\section{Introduction à Plain TeX}\label{introduction-uxe0-plain-tex}

\subsection{Prérequis}\label{pruxe9requis}

Connaître LaTeX.

\section{TeX vs LaTeX}\label{tex-vs-latex}

LaTeX est bien différent de TeX.

Pour être compilable par LaTeX, un fichier doit contenir au minimum
l'instruction de classe de document un groupe document :

\begin{verbatim}
\documentclass{article}
\begin{document}
Hello, world !
\end{document}
\end{verbatim}

En ce qui concerne TeX, le fichier est plus simple :

\begin{verbatim}
Hello, world !
\bye
\end{verbatim}

La seule commande nécessaire est \texttt{\textbackslash{}bye} qui clôt
lance la construction du fichier final au format DVI.

Nombreuses sont les habitudes d'un utilisateur de LaTeX qui ne
fonctionnent pas avec TeX :

\begin{itemize}
\tightlist
\item
  Les listes
  \texttt{\textbackslash{}begin\{enumerate\}...\textbackslash{}end\{enumerate\}}
\item
  Le chapitrage \texttt{\textbackslash{}section\{...\}},
  \texttt{\textbackslash{}subsection\{...\}}
\item
  Le tailles de police \texttt{\textbackslash{}Huge},
  \texttt{\textbackslash{}large}, \texttt{\textbackslash{}small}
\end{itemize}

Avec TeX, il faut tout gérer à la main : créer des compteurs, charger
des fontes, etc.

En clair pour le travail de rédaction usuel, inutile de s'intéresser aux
particularités de TeX, les macros LaTeX sont infiniment plus pratiques.

Pourquoi faudrait-il alors se plonger dans l'étude de la version
originale ? Parce que TeX propose un véritable langage de programmation
avec des commandes permettant de travailler avec des variables, de
construire des structures conditionnelles des boucles et aussi des
procédures.

\subsection{Registres ou Variables}\label{registres-ou-variables}

Pour stocker des valeurs, TeX utilise une notion proche de celle de
variable typée nommée registre. Nous nous intéressons ici à deux types
de registres

\begin{itemize}
\tightlist
\item
  \texttt{count} pour des nombres entiers
\item
  \texttt{dimen} pour des nombres réels avec dimension
\end{itemize}

Pour son fonctionnement, TeX utilise évidemment de nombreux registres.
Par exemple pour définir l'espacement entre les lignes, la taille des
marges ou encore les numéros de page. Il fait donc être prudent quand on
choisit de modifier un registre. Le plus simple est d'en créer soi-même
avec des noms sans équivoque

\begin{verbatim}
\newcount\annee         % défnit le compteur \annee
\annee=2017             % assigne 2017 au compteur
\newdimen\valpi         % définit la dimension \valpi
\valpi=3.14pt           % assigne 3.14pt à la dimension

En \number\annee, $\pi$ s'approche de \the\valpi.
% renvoie
% En 2017, π s'approche de 3.14pt.
\end{verbatim}

\begin{longtable}[c]{@{}rllc@{}}
\caption{Demonstration of pipe table syntax.}\tabularnewline
\toprule
Right & Left & Default & Center\tabularnewline
\midrule
\endfirsthead
\toprule
Right & Left & Default & Center\tabularnewline
\midrule
\endhead
12 & 12 & 12 & 12\tabularnewline
123 & 123 & 123 & 123\tabularnewline
1 & 1 & 1 & 1\tabularnewline
\bottomrule
\end{longtable}
